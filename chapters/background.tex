\chapter{\ifenglish Background Knowledge and Theory\else ทฤษฎีที่เกี่ยวข้อง\fi}

\quad การจัดทําโครงงานพัฒนาเกมนี้ ซึ่งเป็นเกม RTS ที่ควบคุมการเล่นผ่านทางแป้นพิมพ์ และเมาส์ เบื้องต้นในการพัฒนาโปรแกรมนั้นได้ศึกษาค้นคว้าเกี่ยวกับ Game engine ที่ใช้ในการพัฒนาเกม 
และศึกษาเกี่ยวกับประเภทของเกม เนื้อหาในบทนี้จะอธิบายส่วนต่างๆที่เกี่ยวข้องกับโครงงาน เพื่อเพิ่มความเข้าใจในบทถัดไปได้ง่ายมากยิ่งขึ้น

\section{Unity (game engine)}

\qquad Unity คือซอฟต์แวร์สําหรับใช้เพื่อสําหรับการพัฒนาซอฟต์แวร์และการจําลองต่างๆ เป็นโปรแกรมที่มีความสามารถหลากหลาย ได้แก่ การสร้างเกม 2 มิติ, 3 มิติ การสร้าง AR, VR 
สามารถส่งแอพพลิเคชั่นได้ทั้งระบบ Windows, iOS และ Android

\enskip ในการเขียนโปรแกรมเกมนั้น Unity สามารถเขียนด้วยภาษา C++ ได้ แต่ยังไงก็ตามผู้พัฒนาสามารถเขียนภาษา C++ ได้แค่ในส่วนการทํางานพื้นหลังเท่านั้น เช่น ระบบเสียง, บริการพื้นหลัง, ปลั๊กอิน ซึ่งการ
ปฏิสัมพันธ์กับ Engine ทําได้ค่อนข้างน้อย ดังนั้นภาษา C\# จึงเหมาะสมสําหรับกับการใช้ในการพัฒนามากกว่า


\section{เกมกลยุทธ์}
\subsection{เกมกลยุทธ์แบบเรียลไทม์ (Real-Time Strategy : RTS)}

\qquad เป็นประเภทย่อยของวิดีโอเกมกลยุทธ์ (Strategy game) เกมกลยุทธ์แบบเรียลไทม์ช่วยให้ผู้เล่นทุกคนสามารถเล่นเกมแบบ “Real-time” ได้พร้อมกัน

\enskip เป็นเกมที่มุ่งเน้นไปที่การใช้ทรัพยากรเพื่อสร้าง unit และเอาชนะคู่ต่อสู้ เกมกลยุทธ์แบบเรียลไทม์มักจะถูกเปรียบเทียบกับเกมวางแผนแบบ turn based ซึ่งผู้เล่นแต่ละคนมีเวลาพิจารณาอย่างรอบคอบในการเคลื่อนที่ครั้งต่อไปโดยไม่ต้องกังวลเกี่ยวกับการกระทําของฝ่ายตรงข้าม 
แต่ในเกมกลยุทธ์แบบเรียลไทม์ผู้เล่นจะต้องพยายามปกป้องฐานและเริ่มการโจมตีในขณะที่รู้ว่าฝ่ายตรงข้ามกําลังต่อสู้เพื่อทําสิ่งเดียวกัน

\subsection{การเล่นเกม}

\qquad ในเกมกลยุทธ์แบบเรียลไทม์ทั่วไปหน้าจอจะแบ่งออกเป็นพื้นที่แผนที่ที่แสดงโลกของเกม, ภูมิประเทศและสิ่งปลูกสร้าง ผู้เล่นมักจะได้เล่นในมุมมองจากที่สูง 
ควบคุมโดยการเลื่อนหน้าจอและออกคําสั่งด้วยเมาส์และอาจใช้แป้นพิมพ์ลัด (short key)

\enskip การเล่นเกมโดยทั่วไปประกอบด้วยผู้เล่นที่อยู่ในตําแหน่งใดที่หนึ่งในแผนที่โดยมี unit ไม่กี่ unit หรือ building ที่สามารถสร้าง unit / building อื่นได้ 
โดยมักจะให้ผู้เล่นสร้างสิ่งก่อสร้างที่เฉพาะเจาะจงเพื่อปลดล็อค unit ในขั้นที่สูงขึ้น เกม RTS ต้องการให้ผู้เล่นสร้างกองทัพ 
(ตั้งแต่ unit เล็ก ๆ ไม่เกิน 2 unit ไปจนถึง unit หลายร้อย unit) ใช้เพื่อป้องกันตัวเองจากการโจมตีของศัตรูหรือกําจัดศัตรูที่ครอบครองฐานที่มีกําลังการผลิต unit เป็นของตัวเอง 
ในบางครั้งเกม RTS จะมีจํานวน unit ที่กําหนดไว้ล่วงหน้าเพื่อให้ผู้เล่นควบคุมและไม่อนุญาตให้สร้าง unit เพิ่มเติมเกินจากที่กําหนดไว้

% \section{Third section}
% Section 3 text. The dielectric constant\index{dielectric constant}
% at the air-metal interface determines
% the resonance shift\index{resonance shift} as absorption or capture occurs
% is shown in Equation~\eqref{eq:dielectric}:

% \begin{equation}\label{eq:dielectric}
% k_1=\frac{\omega}{c({1/\varepsilon_m + 1/\varepsilon_i})^{1/2}}=k_2=\frac{\omega
% \sin(\theta)\varepsilon_\mathit{air}^{1/2}}{c}
% \end{equation}

% \noindent
% where $\omega$ is the frequency of the plasmon, $c$ is the speed of
% light, $\varepsilon_m$ is the dielectric constant of the metal,
% $\varepsilon_i$ is the dielectric constant of neighboring insulator,
% and $\varepsilon_\mathit{air}$ is the dielectric constant of air.

% \section{About using figures in your report}

% define a command that produces some filler text, the lorem ipsum.
% \newcommand{\loremipsum}{
%   \textit{Lorem ipsum dolor sit amet, consectetur adipisicing elit, sed do
%   eiusmod tempor incididunt ut labore et dolore magna aliqua. Ut enim ad
%   minim veniam, quis nostrud exercitation ullamco laboris nisi ut
%   aliquip ex ea commodo consequat. Duis aute irure dolor in
%   reprehenderit in voluptate velit esse cillum dolore eu fugiat nulla
%   pariatur. Excepteur sint occaecat cupidatat non proident, sunt in
%   culpa qui officia deserunt mollit anim id est laborum.}\par}

% \begin{figure}
%   \centering

%   \fbox{
%      \parbox{.6\textwidth}{\loremipsum}
%   }

%   % To include an image in the figure, say myimage.pdf, you could use
%   % the following code. Look up the documentation for the package
%   % graphicx for more information.
%   % \includegraphics[width=\textwidth]{myimage}

%   \caption[Sample figure]{This figure is a sample containing \gls{lorem ipsum},
%   showing you how you can include figures and glossary in your report.
%   You can specify a shorter caption that will appear in the List of Figures.}
%   \label{fig:sample-figure}
% \end{figure}

% Using \verb.\label. and \verb.\ref. commands allows us to refer to
% figures easily. If we can refer to Figures
% \ref{fig:walrus} and \ref{fig:sample-figure} by name in the {\LaTeX}
% source code, then we will not need to update the code that refers to it
% even if the placement or ordering of the figures changes.

% \loremipsum\loremipsum

% This code demonstrates how to get a landscape table or figure. It
% uses the package lscape to turn everything but the page number into
% landscape orientation. Everything should be included within an
% \afterpage{ .... } to avoid causing a page break too early.
% \afterpage{
%   \begin{landscape}
%   \begin{table}
%     \caption{Sample landscape table}
%     \label{tab:sample-table}

%     \centering

%     \begin{tabular}{c||c|c}
%         Year & A & B \\
%         \hline\hline
%         1989 & 12 & 23 \\
%         1990 & 4 & 9 \\
%         1991 & 3 & 6 \\
%     \end{tabular}
%   \end{table}
%   \end{landscape}
% }

% \loremipsum\loremipsum\loremipsum

% \section{Overfull hbox}

% When the \verb.semifinal. option is passed to the \verb.cpecmu. document class,
% any line that is longer than the line width, i.e., an overfull hbox, will be
% highlighted with a black solid rule:
% \begin{center}
% \begin{minipage}{2em}
% juxtaposition
% \end{minipage}
% \end{center}

\section{\ifenglish%
\ifcpe CPE \else ISNE \fi knowledge used, applied, or integrated in this project
\else%
ความรู้ตามหลักสูตรซึ่งถูกนำมาใช้หรือบูรณาการในโครงงาน\fi}

\qquad ในการจัดทำโครงงานนี้ มีหลายหัวข้อที่ผู้พัฒนานั้นยังไม่มีความชำนาญ และยังขาดประสบการณ์ 
จึงได้นำความรู้ที่ได้รับจากการเรียนการสอนตามหลักสูตรของวิศวกรรมคอมพิวเตอร์ มหาวิทยาลัยเชียงใหม่มาใช้ ดังนี้
\begin{enumerate}
  \item Object-Oriented Programming (OOP)
  
  \qquad ใช้ช่วยในการพัฒนาโปรแกรม,การเขียนโปรแกรมเชิงวัตถุ ทำให้ object ต่างๆภายในเกมสามารถทำงานได้อย่างถูกต้องและมีประสิทธิภาพ
  \item Algorithms
  
  \qquad ช่วยกำหนดทิศทางและเป้าหมายในพัฒนาโปรแกรม วางแผนแก้ไขปัญหาอย่างเป็นขั้นตอน
  \item Software Engineering
  
  \qquad ช่วยทำให้การพัฒนาซอฟต์แวร์เกิดขึ้นอย่างเป็นระบบ
  \item Human-computer interaction (HCI) 
  
  \qquad ช่วยในการออกแบบโปรแกรม เพื่อสนับสนุนประสบการณ์การใช้งานของผู้ใช้ (user eperience)
\end{enumerate}

\section{\ifenglish%
Extracurricular knowledge used, applied, or integrated in this project
\else%
ความรู้นอกหลักสูตรซึ่งถูกนำมาใช้หรือบูรณาการในโครงงาน\fi}

\qquad ในการจัดทำโครงงานนี้ นอกเหนือจากความรู้ตามหลักสูตรของทางมหาวิทยาลัย ผู้พัฒนายังได้ไปศึกษาเนื้อหานอกหลักสูตรเพื่อนำมาใช้ในการพัฒนาโปรแกรม ดังนี้

\begin{enumerate}
  \item MDA framework 
  
  \qquad ช่วยในการออกแบบเกม เพื่อให้ผู้เล่นเกิดความสนุกไปกับเกม
  \item การพัฒนาเกมด้วย Unity engine
  
  \qquad เพื่อช่วยอำนวยความสะดวกให้กับการพัฒนาเกม
\end{enumerate}
